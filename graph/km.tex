\subsection{KM}
    还是hehe的版\\
    配合华华的KM看吧。\\
    \begin{lstlisting}[language=c++]
int w[16][16];
int l[16];
int r[16];
int low[16];
int n;
int flag1[16];
int flag[16];
int f[16];
int qw[16];
const int INF=10000000;
int ans;
int dfs(int x)
{
	flag1[x]=1;
	int i;
	for(i=1;i<=n;i++)
	{
		if(flag[i]==0&&w[x][i]==l[x]+r[i])
		{
			flag[i]=1;
			if(f[i]==0||dfs(f[i]))
			{
				f[i]=x;
				return 1;
			}
		}
		low[i]=min(low[i],w[x][i]-l[x]-r[i]);
//(l[x]+r[i]-w[x][i]最大匹配)
	}
	return 0;
}
int km(void)
{
	memset(f,0,sizeof(f));
	memset(r,0,sizeof(r));
	int i;
	for(i=1;i<=n;i++)
	{
		int j;
		int mi=INF;
		for(j=1;j<=n;j++)
		{
			if(w[i][j]<mi)
				mi=w[i][j];
		}
		l[i]=mi;
	}
//(赋值为边权最大值。。最大匹配)
	for(i=1;i<=n;i++)
	{
		while(1)
		{
			memset(flag,0,sizeof(flag));
			memset(flag1,0,sizeof(flag1));
			int j;
			for(j=1;j<=n;j++)
				low[j]=INF;
			if(dfs(i))
				break;
			int d=INF;
			for(j=1;j<=n;j++)
			{
				if(flag[j]==0)
				{
					d=min(d,low[j]);
				}
			}
			for(j=1;j<=n;j++)
			{
				if(flag1[j])
					l[j]+=d;(-为最大匹配)
				if(flag[j])
					r[j]-=d;(+为最大匹配)
			}
		}
	}
	int sum=0;
	int j;
	for(j=1;j<=n;j++)
	{
		sum+=l[j];
		sum+=r[j];
	}
	return sum;
}
    \end{lstlisting}
